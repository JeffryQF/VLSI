%%%%%%%%%%%%%%%%%%%%%%%%%%%%%%%%%%%%%%%%%
% Engineering Calculation Paper
% LaTeX Template
% Version 1.0 (20/1/13)
%
% This template has been downloaded from:
% http://www.LaTeXTemplates.com
%
% Original author:
% Dmitry Volynkin (dim_voly@yahoo.com.au)
%
% License:
% CC BY-NC-SA 3.0 (http://creativecommons.org/licenses/by-nc-sa/3.0/)
%
% Modificaciones por Roberto Cerdas
%
% Si desea utilizar notas al margen, favor leer los comentarios en las líneas 32 y % 52. Si desea colocar un logo, favor leer comentario en línea 54. El comando     % \marginnote{texto} introduce notas al margen.  
%
%%%%%%%%%%%%%%%%%%%%%%%%%%%%%%%%%%%%%%%%%

%----------------------------------------------------------------------------------------
%	PACKAGES AND OTHER DOCUMENT CONFIGURATIONS
%----------------------------------------------------------------------------------------

\documentclass[12pt,a4paper]{article} % Use A4 paper with a 12pt font size - different paper sizes will require manual recalculation of page margins and border positions

\usepackage[spanish]{babel} % Utilizar reglas de idioma español
\usepackage[utf8]{inputenc} % Use UTF-8 encoding
\usepackage{marginnote} % Required for margin notes
\usepackage{wallpaper} % Required to set each page to have a background
\usepackage{lastpage} % Required to print the total number of pages
%\usepackage[left=1.3cm,right=4.6cm,top=1.8cm,bottom=4.0cm,marginparwidth=3.4cm]{geometry} % Comentar la línea abajo y descomentar esta para usar notas al margen
\usepackage[left=1.3cm,right=1.3cm,top=1.8cm,bottom=4.0cm]{geometry} % Adjust page margins
\usepackage{amsmath} % Required for equation customization
\usepackage{amssymb} % Required to include mathematical symbols
\usepackage{xcolor} % Required to specify colors by name
\usepackage[square, comma, sort&compress]{natbib} % Use the natbib reference package - read up on this to edit the reference style; if you want text (e.g. Smith et al., 2012) for the in-text references (instead of numbers), remove 'numbers' 

\usepackage{fancyhdr} % Required to customize headers
\setlength{\headheight}{80pt} % Increase the size of the header to accommodate meta-information
\pagestyle{fancy}\fancyhf{} % Use the custom header specified below
\renewcommand{\headrulewidth}{0pt} % Remove the default horizontal rule under the header

\setlength{\parindent}{0cm} % Remove paragraph indentation
\newcommand{\tab}{\hspace*{2em}} % Defines a new command for some horizontal space

\newcommand\BackgroundStructure{ % Command to specify the background of each page
\setlength{\unitlength}{1mm} % Set the unit length to millimeters

\setlength\fboxsep{0mm} % Adjusts the distance between the frameboxes and the borderlines
\setlength\fboxrule{0.5mm} % Increase the thickness of the border line
\put(10, 10){\fcolorbox{black}{white!10}{\framebox(192,247){}}} % Main content box
%\put(165, 10){\fcolorbox{black}{blue!10}{\framebox(37,247){}}} % Margin box: Descomentar para utilizar notas al margen.
\put(10, 262){\fcolorbox{black}{white!10}{\framebox(192, 25){}}} % Header box
%\put(143, 263){\includegraphics[height=23mm,keepaspectratio]{logo}} % Logo box - maximum height/width: 25x42. Descomentar esta línea para usar logo.
}

%----------------------------------------------------------------------------------------
%	HEADER INFORMATION
%----------------------------------------------------------------------------------------

\fancyhead[L]{\begin{tabular}{l r | l r} % The header is a table with 4 columns
\textbf{Proyecto} & Diseño función lógica CMOS & \textbf{Página} & \thepage/\pageref{LastPage} \\ % Project name and page count
\textbf{Trabajo} & Proceso de diseño & \textbf{Actualizado en:} & 08/10/2015 \\ % Job number and last updated date
\textbf{Curso} & VLSI & \textbf{Revisado en:} & 09/10/2015 \\ % Version and reviewed date
\textbf{Diseñador} & López F. - Quirós.J.& \textbf{Revisado por:} & Alfonso Chacón Rodríguez \\ % Designer and reviewer
\end{tabular}}

%----------------------------------------------------------------------------------------

\begin{document}

\AddToShipoutPicture{\BackgroundStructure} % Set the background of each page to that specified above in the header information section

%----------------------------------------------------------------------------------------
%	DOCUMENT CONTENT
%----------------------------------------------------------------------------------------


\section{Resumen.} 

En este documento se presentan los cálculos de los tiempos de propagación y contaminación de la compuerta compuesta
\textit{F=(A+B)(C+D)}, obteniendo estos tiempos de manera analítica, mediante la teoría de esfuerzo lógico y la aproximación de Elmore, así como mediante simulación, utilizando los software \textit{Electric} y \textit{LTSpice}, y por último contrastando los resultados de ambos métodos. Tambien se muestra el diseño de un trazado que usa una única tira de difusión en ambos pozos, construyendo los \textit{caminos de Euler}, y dibujando el diagrama de palitos que muestra el orden de entradas, poly, las difusiones \textit{n} y \textit{p}, el pozo y las líneas de metal.\\

\section{Introducción.} 

Para el diseño de una compuerta de lógica CMOS o un conjunto de las mismas, es necesario tomar varias consideraciones para los anchos de canal de los transistores, ya sean para una optimización en potencia o para una rápida conmutación entre los niveles lógicos. En estos se debe incluir la carga que debe soportar  para que los tiempos de retardo no afecten el comportamiento ideal del circuito.\\

Para la demostración de los pasos de diseño de una función lógica, se nos ha pedido diseñar el esquemático y el layout de la función lógica del problema 9.4 de [1] que es \textit{F=(A+B)(C+D)}. El esquemático deberá realizarse con lógica CMOS estática.\\

Luego se procederá a calcular el retardo de propagación y contaminación que presenta la función, tanto de manera analítica con la aproximación de Elmore como con la teoría de esfuerzo lógico, que se evaluará con una simulación en los software \textit{Electric} y \textit{LTSpice} para saber si con este método se puede llegar a una aproximación bastante certera de los tiempos de retardo.\\

Por último, se realizada el layout del circuito y se volverá a realizar la simulación para definir los \textit{pitch} que se usarán en la proxima tarea.\\

\section{Resultados Experimentales.}


\subsection{Diseño del Trazado de una Compuerta y Diagrama de Palitos.}


La función lógica a la que se le realizara el trazado y el \texti{Diagrama de Palitos} es la mostrada en la \textit{Ec.\ref{eqn:logica}}. Esta función lógica es conocida como OAI-22 + INV. Para el diseño de la compuerta se utilizó algebra booleana y se obtuvo la \textit{Ec.\ref{eqn:logica inversa}}.\\

\begin{equation}\label{eqn:logica}
F=(A+B)*(C+D)
\end{equation}

\begin{equation}\label{eqn:logica inversa}
\overline{F}=\overline{(A+B)}+\overline{(C+D)}
\end{equation}\\


Para lo cual se hizo el esquemático mostrado en la \textit{Fig.\ref{fig:Comp_Transistores}} y se hizo el \textit{Diagrama de Palitos} mostrado en la \textit{Fig.\ref{fig:inv_est}}. En este diagrama se dibujaron tanto la compuerta \textit{OAI22} como el \textit{Inversor} para cambiar el nivel lógico a la salida, por lo que se observa mas de una tira de difusión en la parte inferior, resultado de la inclusión del \textit{Inversor} en el diagrama.\\

\begin{figure}[htbp]
  \centering
    \includegraphics[scale=0.4]{./Palitos.png}
    \rule{35em}{0.5pt}
  \caption[IdealvsSim]{Diagrama de palitos de la función OAI22 + INV}
  \label{fig:inv_est}
\end{figure}


\subsection{Delay de Propagación y de Contaminación.}

\subsubsection{Método Analítico}

El cálculo de los tiempos de propagación y de contaminación se realiza haciendo uso de dos métodos: la teoría del esfuerzo lógico, \textit{logical effort}, y por el método de la aproximación de Elmore, \textit{Elmore Delay}.\\

\subsubsection{Método de Esfuerzo Lógico}

El esfuerzo lógico se define como \textit{"la razón de la capacitancia del gate a la capacitancia de entrada de un inversor que puede entregar la misma corriente de salida."}, e indica que tan mala es una compuerta produciendo una corriente de salida comparada con un inversor.\\

Para el cálculo del delay por medio de la teoría de esfuerzo lógico, se utilizan las formulas del cálculo del delay en redes lógicas con multiples etapas, \textit{Multistage Logical Network}, que utiliza las siguientes formulas para el cálculo del delay:\\

\begin{equation}\label{eqn:esfuerzo_logico}
G= \prod g_{i}
\end{equation}

\begin{equation}\label{eqn:esfuerzo_electrico}
H= \frac{C_{out-path}}{C_{in-path}}
\end{equation}

\begin{equation}\label{eqn:esfuerzo_enramado}
B= \prod b_{i}
\end{equation}

\begin{equation}\label{eqn:esfuerzo}
F = GBH
\end{equation}

\begin{equation}\label{eqn:delay_parasitico}
P = \sum p_{i}
\end{equation}

\begin{equation}\label{eqn:delay}
D = NF^{\frac{1}{N}} + P
\end{equation}

En donde \textit{G} es el esfuerzo lógico, \textit{H} es el esfuerzo eléctrico, \textit{B} el esfuerzo de enramado, \textit{F} es el esfuerzo total, \textit{P} es el delay parasítico del camino, \textit{N} es la cantidad de estapas del camino y \textit{D} es el delay total del camino.\\

La función \textit{F=(A+B)(C+D)} se puede representar a nivel de compuerta como se muestra en la \textit{Fig.\ref{fig:OAI21}} , donde se observa que la compuerta es del tipo OR-OR-AND-INV + INVERSOR, \textit{OAI-21 + inverter}, y a partir de aqui se calcula el esfuerzo lógico de camino.\\

\begin{figure}[htbp]
  \centering
    \includegraphics[scale=0.5]{./OAI21.png}
    \rule{35em}{0.5pt}
  \caption[IdealvsSim]{Compuerta \textit{OAI-21}.}
  \label{fig:OAI21}
\end{figure}

Para realizar los cálculos del esfuerzo de camino de esta función, se debe tomar en cuenta que cada entrada presenta como maximo 30$\lambda$ de ancho de transistor, y que la salida debe de manejar una carga equivalente de 500$\lambda$ de ancho de transistor, como se muestra en la \textit{Fig.\ref{fig:OAI21_Cargas}}. Se puede observar que en este caso el número de etapas, \textit{N}, es igual a 2, por lo que haciendo uso de las ecuaciones \textit{\ref{eqn:esfuerzo_logico}, \ref{eqn:esfuerzo_electrico}, \ref{eqn:esfuerzo_enramado}, \ref{eqn:esfuerzo}, \ref{eqn:delay_parasitico}, \ref{eqn:delay}}, se puede encontrar el delay mediante la teoría de esfuerzo lógico de cada entrada bajo estas condiciones de carga.\\

\begin{figure}[htbp]
  \centering
    \includegraphics[scale=0.4]{./OAI21_Cargas.png}
    \rule{35em}{0.5pt}
  \caption[IdealvsSim]{Compuerta \textit{OAI-21} con Carga.}
  \label{fig:OAI21_Cargas}
\end{figure}


\begin{equation}\label{eqn:esfuerzo_logico2}
G= \prod g_{i}= \frac{6}{3} * 1 = \frac{6}{3}
\end{equation}

\begin{equation}\label{eqn:esfuerzo_electrico2}
H= \frac{C_{out-path}}{C_{in-path}} = \frac{500\lambda}{30\lambda} = \frac{50}{3}
\end{equation}

\begin{equation}\label{eqn:esfuerzo_enramado2}
B= \prod b_{i} = 1
\end{equation}

\begin{equation}\label{eqn:esfuerzo2}
F = GBH = \frac{6}{3}*1*\frac{50}{3} = \frac{100}{3}
\end{equation}

\begin{equation}\label{eqn:delay_parasitico2}
P = \sum p_{i} = \frac{12}{3} + 1 = \frac{15}{3} = 5
\end{equation}

\begin{equation}\label{eqn:delay2}
D = NF^{\frac{1}{N}} + P = 2*(\frac{100}{3})^{\frac{1}{2}} + 5 = 16.54\tau
\end{equation}\\

El cálculo de los tiempos arroja como resultado que cada entrada de esta compuerta tendrá un delay de \textit{16.54$\tau$}.\\

Con el valor de delay calculado se puede proceder a dimensionar los transistores que conforman la compuerta. En la \textit{Fig.\ref{fig:Comp_Transistores}} se muestra la compuerta compuesta a nivel de transistores, sin dimensionar.\\

\begin{figure}[htbp]
  \centering
    \includegraphics[scale=0.45]{./Comp_Transistores.png}
    \rule{35em}{0.5pt}
  \caption[IdealvsSim]{Esquemático de la Compuerta \textit{F=(A+B)(C+D)} a nivel de transistores sin dimensionar.}
  \label{fig:Comp_Transistores}
\end{figure}

Para dimensionar los transistores de cada entrada, se toma en cuenta el modelo RC del transistor. Sabiendo que la resitencia de la red \textit{PMOS} debe ser igual a la de la red \textit{NMOS}, y que cada entrada presenta como maximo 30$\lambda$ de ancho de transistor se obtiene que:\\

\begin{equation}\label{eqn:R}
\frac{2R}{k_p} = \frac{R}{k_n}
\end{equation}

\begin{equation}\label{eqn:k}
k_p + k_n = 30\lambda
\end{equation}\\

Donde \textit{$k_{p}$} y \textit{$k_{n}$} son los anchos de los tansistores \textit{p} y \textit{n} de cada entrada. Con este sistema de ecuaciones se obtiene que:

\begin{equation}\label{eqn:R1}
k_p = 2k_n
\end{equation}

\begin{equation}\label{eqn:k1}
k_p + k_n = 30\lambda
\end{equation}

\begin{equation}\label{eqn:k2}
3k_n = 30\lambda
\end{equation}

\begin{equation}\label{eqn:k3}
k_n = 10\lambda ; k_p = 20\lambda
\end{equation}\\

Utilizando la \textit{Ec.\ref{eqn:k_Inv}}, se puede encontrar la capacitancia de entrada del inversor, como se muestra a continuación:\\

\begin{equation}\label{eqn:k_Inv}
C_{ini}= \frac{C_{outi}*g_{i}}{F^{\frac{1}{N}}}
\end{equation}

\begin{equation}\label{eqn:k_Inv2}
{C_{in}}_{i}= \frac{500\lambda*1}{5.777} = 86.6\lambda = 87\lambda
\end{equation}\\

Sabiendo que la resistencia en la red de \textit{pull-up} es el doble que la de la red de \textit{pull-down} en el inversor de la salida, se puede encontrar las dimensiones de los anchos de los transistores en dicho inversor.\\

\begin{equation}\label{eqn:k_Inv3}
k_p=2k_n
\end{equation}

\begin{equation}\label{eqn:k_Inv4}
3k_n=87\lambda
\end{equation}

\begin{equation}\label{eqn:k_Inv5}
k_n=29\lambda ; k_p=58\lambda 
\end{equation}\\

En la \textit{Fig.\ref{fig:Comp_Transistores_Dim}} se muestra la compuerta a nivel de transistores, cada uno con su respectivo valor de ancho de canal correspondiente.\\

\begin{figure}[htbp]
  \centering
    \includegraphics[scale=0.45]{./Comp_Transistores_Dim.png}
    \rule{35em}{0.5pt}
  \caption[Delay]{Esquemático de la Compuerta \textit{F=(A+B)(C+D)} a nivel de transistores con sus respectivas dimensiones, \textit{L=2$\lambda$}.}
  \label{fig:Comp_Transistores_Dim}
\end{figure}

\subsubsection{Método de Aproximación de Elmore}

Ahora se calcula el delay de propagación y el delay de contaminación de la compuerta \textit{F=(A+B)(C+D)} mediante el método de \textit{Aproximación de Elmore}, el cuál se hace valer del modelo RC del transistor, para calcular los delays de una compuerta. El modelo de delay de \textit{Elmore} estima el restraso desde una fuente conmutando a uno de los nodos hoja cambiantes como la suma sobre cada nodo \textit{i} de la capacitancia \textit{$C_{i}$} en el nodo, multiplicado por la resistencia efectiva \textit{$R_{is}$} en el camino compartido desde la fuente hasta el nodo y la hoja, dando como resultado la ec.\ref{eqn:Elmore}:


\begin{equation}\label{eqn:Elmore}
t_{pd} = \sum_{i} R_{is}C_{i}
\end{equation}\\



En la \textit{Fig.\ref{fig:ModeloRC_Completo}} se muestra el modelo \textit{RC} completo de la compuerta, con sus capacitancias y resistencias, para la compuerta \textit{OAI-21} y para el inversor en la salida.\\

\begin{figure}[htbp]
  \centering
    \includegraphics[scale=0.45]{./ModeloRC_Completo.png}
    \rule{35em}{0.5pt}
  \caption[IdealvsSim]{Modelo \textit{RC} completo de la Compuerta \textit{F=(A+B)(C+D)}.}
  \label{fig:ModeloRC_Completo}
\end{figure}

Para este modelo se calculan el mejor y el peor caso para el delay, por lo que en las siguientes secciones se presentaran los cálculos y los circuitos \textit{RC} de cada caso.\\

\subsubsection{Red de Pull-Up}

Antes de calcular el mejor y el pesor caso de subida, primero se calcula el retraso provocado por el inversor a la salida, ya que este sera el mismo para el mejor y el peor caso de subida y de bajada. En la \textit{Fig.\ref{fig:RC_Inversor}} se muestra el modelo \textit{RC} del inversor de este problema.\\

\begin{figure}[htbp]
  \centering
    \includegraphics[scale=0.3]{./RC_Inversor.png}
    \rule{35em}{0.5pt}
  \caption[IdealvsSim]{Modelo \textit{RC} del Inversor.}
  \label{fig:RC_Inversor}
\end{figure}

Con este circuito se calcula el delay utilizando la \textit{Ec.\ref{eqn:Elmore_Inv}}, obteniendo como resultado que el delay del inversor para el caso de subida y bajada es:\\

\begin{equation}\label{eqn:Elmore_Inv}
t_{pd} = \frac{R}{29}(500 + 87)C = \frac{587RC}{29} = 20.24RC
\end{equation}\\

Con el delay del inversor calculado en la \textit{Ec.\ref{eqn:Elmore_Inv}}, se procede a realizar los calculos para el mejor y peor caso de delay de subida.\\

En la \textit{Fig.\ref{fig:Mejor_caso_Rise_propagacion}} se observa que el mejor caso ocurre cuando todos los transistores de la red de \textit{pull-up} se encuentran encendidos, o dos transistores están encendidos en la red de \textit{pull-up}, A y B, y los dos transistores mas interiores de la red de \textit{pull-down}, A y B, están apagados, con lo que la red de \textit{pull-down} no contribuye al delay en ambos casos. En la \textit{Ec.\ref{eqn:Elmore_tpdr_Mejor}} se muestra el cálculo del delay para el mejor caso de levantamiento:\\

\begin{figure}[htbp]
  \centering
    \includegraphics[scale=0.5]{./Mejor_caso_Rise_propagacion.png}
    \rule{35em}{0.5pt}
  \caption[IdealvsSim]{Mejor Caso de \textit{Rise} en el Delay de Propagación.}
  \label{fig:Mejor_caso_Rise_propagacion}
\end{figure}

\begin{equation}\label{eqn:Elmore_tpdr_Mejor}
t_{pdr} = \frac{20RC}{10} + \frac{147*2RC}{10} + t_{pdInv}= 31.4RC + 20.24RC = 51.64RC = 17.213\tau
\end{equation}\\

En la \textit{Fig.\ref{fig:Peor_caso_Rise_propagacion}} se observa que el peor caso ocurre cuando solo dos transistores se encendidos, C y D, y uno o los dos transistores mas interiores de la red de \textit{pull-down} están también encendidos, por lo que la red de \textit{pull-down} contribuye al delay. En la \textit{Ec.\ref{eqn:Elmore_tpdr_Peor}} se muestra el cálculo del delay para el peor caso de levantamiento:\\

\begin{figure}[htbp]
  \centering
    \includegraphics[scale=0.45]{./Peor_caso_Rise_propagacion.png}
    \rule{35em}{0.5pt}
  \caption[IdealvsSim]{Peor Caso de \textit{Rise} en el Delay de Propagación.}
  \label{fig:Peor_caso_Rise_propagacion}
\end{figure}

\begin{equation}\label{eqn:Elmore_tpdr_Peor}
t_{pdr} = \frac{20RC}{10}+\frac{20RC}{5}+\frac{147RC}{5}+t_{pdInv}= 35.4RC + 20.24RC = 55.64RC = 18.546\tau
\end{equation}\\

\subsubsection{Red de Pull-Down}

En la \textit{Fig.\ref{fig:Peor_caso_Fall_propagacion}} se observa el peor caso de caída en el cálculo del delay en la compuerta. En este caso se activan las entradas A y C, con un 1 lógico, en la red de \textit{pull-down}, y se activan las entradas B y D, con un 0 lógico, en la red de \textit{pull-up}, con lo que esta red contribuye con el delay, tal como se muestra en la \textit{Fig.\ref{fig:Peor_caso_Fall_propagacion}}. En la \textit{Ec.\ref{eqn:Delay_fall_Peor}} se muestra el cálculo del delay.\\

\begin{figure}[htbp]
  \centering
    \includegraphics[scale=0.25]{./Peor_caso_Fall_propagacion.png}
    \rule{35em}{0.3pt}
  \caption[C_Carga]{Peor Caso de \textit{Fall} en el Delay de Propagación.}
  \label{fig:Peor_caso_Fall_propagacion}
\end{figure}


\begin{equation}\label{eqn:Delay_fall_Peor}
t_{pdf} = \frac{20RC}{10}+\frac{147RC}{5}+\frac{40RC}{5}+t_{pdInv}=39.4RC+20.24RC=59.64RC=19.88\tau
\end{equation}\\


En la \textit{Fig.\ref{fig:Mejor_caso_Fall_propagacion}} se observa el mejor caso de caída en el cálculo del delay en la compuerta. En este caso se activan las entradas B y D, con un 1 lógico, en la red de \textit{pull-down}, y por ende se desactivan las entradas B y D en la red de \textit{pull-up}, con lo que esta red no contribuye con el delay, tal como se muestra en la \textit{Fig.\ref{fig:Mejor_caso_Fall_propagacion}}. En la \textit{Ec.\ref{eqn:Delay_fall_Mejor}} se muestra el cálculo del delay.\\

\begin{figure}[htbp]
  \centering
    \includegraphics[scale=0.25]{./Mejor_caso_Fall_propagacion.png}
    \rule{35em}{0.3pt}
  \caption[C_Carga]{Mejor Caso de \textit{Fall} en el Delay de Propagación.}
  \label{fig:Mejor_caso_Fall_propagacion}
\end{figure}


\begin{equation}\label{eqn:Delay_fall_Mejor}
t_{pdf} = \frac{20RC}{10}+\frac{147RC}{5}+t_{pdInv}=31.4RC+20.24RC=51.64RC=17.21\tau
\end{equation}\\


\subsubsection{Delay de Contaminación.}

El delay de contaminación, \textit{contamination delay}, indica que tán rapido puede conmutar la compuerta. Para este delay se procede a calcular el mejor y el peor caso de retraso de \textit{raise} y \textit{fall}.\\

En la \textit{Fig.\ref{fig:Mejor_Caso_Rise_contam}} se muestra el mejor caso de delay de contaminación de levantamiento, que se da cuando todos los transistores de la red de \textit{pull-up} se encienden simultaneamente. En la \textit{Ec.\ref{eqn:DelayC_rise_Mejor}} se muestra el resultado de cacular el delay de contaminación para el mejor caso de levantamiento.\\


\begin{figure}[htbp]
  \centering
    \includegraphics[scale=0.3]{./Mejor_Caso_Rise_contam.png}
    \rule{35em}{0.3pt}
  \caption[C_Carga]{Mejor Caso de \textit{Rise} en el Delay de Contaminación.}
  \label{fig:Mejor_Caso_Rise_contam}
\end{figure}

\begin{equation}\label{eqn:DelayC_rise_Mejor}
t_{cdr} = \frac{147RC}{10}+t_{pcdInv}=14.7RC+\frac{500RC}{29}=31.94RC=10.65\tau
\end{equation}\\

Para el peor caso de delay de contaminación de levantamiento, el peor caso ocurre cuando solo una rama de la red de \textit{pull-up} se encuentra encendida, como se muestra en la \textit{Fig.\ref{fig:Peor_Caso_Rise_contam}}. En la \textit{Ec.\ref{eqn:DelayC_rise_Peor}} se muestra el resultado del cálculo del delay de contaminación.\\


\begin{figure}[htbp]
  \centering
    \includegraphics[scale=0.4]{./Peor_Caso_Rise_contam.png}
    \rule{35em}{0.3pt}
  \caption[C_Carga]{Peor Caso de \textit{Rise} en el Delay de Contaminación.}
  \label{fig:Peor_Caso_Rise_contam}
\end{figure}

\begin{equation}\label{eqn:DelayC_rise_Peor}
t_{cdr} = \frac{147RC}{5}+t_{pcdInv}=29.4RC+\frac{500RC}{29}=46.64RC=15.55\tau
\end{equation}\\


En el caso del delay de contaminacion de caída, el mejor caso corresponde cuando los transistores más externos están encendidos y se encienden los dos más internos. Así la capacitancia del nodo inferior está descargada y no contribuye al delay. El circuito \textit{RC} correspondiente a este caso se observa en la \textit{Fig.\ref{fig:Mejor_Caso_Fall_contam}}, donde ya se han hecho las reducciones de resistencias correspondientes, y en la \textit{Ec.\ref{eqn:DelayC_Fall_Mejor}} se muestran los resultados del cálculo.\\

\begin{figure}[htbp]
  \centering
    \includegraphics[scale=0.25]{./Mejor_Caso_Fall_contam.png}
    \rule{35em}{0.3pt}
  \caption[C_Carga]{Mejor Caso de \textit{Fall} en el Delay de Contaminación.}
  \label{fig:Mejor_Caso_Fall_contam}
\end{figure}

\begin{equation}\label{eqn:DelayC_Fall_Mejor}
t_{cdf} = \frac{147RC}{10}+t_{pcdInv}=14.7RC+\frac{500RC}{29}=31.94RC=10.65\tau
\end{equation}\\


El peor caso para el delay de contaminación de caída, se muestra en la \textit{Fig.\ref{fig:Peor_Caso_Fall_contam}}, en donde solo dos transistores de la red de \textit{pull-down} permiten la descarga de la capacitancia de carga. En la \textit{Ec.\ref{eqn:DelayC_Fall_Peor}} se muestra el cálculo del delay de contaminación de caída en el peor caso.\\

\begin{figure}[htbp]
  \centering
    \includegraphics[scale=0.25]{./Peor_Caso_Fall_contam.png}
    \rule{35em}{0.3pt}
  \caption[C_Carga]{Peor Caso de \textit{Fall} en el Delay de Contaminación.}
  \label{fig:Peor_Caso_Fall_contam}
\end{figure}

\begin{equation}\label{eqn:DelayC_Fall_Peor}
t_{cdf} = \frac{147RC}{5}+t_{pcdInv}=29.4RC+\frac{500RC}{29}=46.4RC=15.55\tau
\end{equation}\\


\subsubsection{Simulación de Retardos de Propagación}\\

A partir de la \textit{Ec.\ref{eqn:logica}} y de las constantes calculadas y encontradas en el informe anterior para el dimensionamiento de los transistores para tiempos de retardo y levantamiento simétrico de los niveles lógicos encontramos que para una carga de entrada de $30\lambda$ y una carga de salida de $500\lambda$ se encontró que el ancho de canal de los transistores equivalentes son los mostrados el la Tabla \ref{table:ancho_canal} y el circuito es el mostrado en la \textit{Fig.\ref{fig:circuito_E}}.\\


\begin{table}\label{table:ancho_canal}
\begin{center}
\begin{tabular}{c||c||c}
  & Ancho nmos ($\lambda$) & Ancho pmos ($\lambda$)\\
\hline
\hline
OAI22 & 11.11 & 18.88 \\
Inv & 32.22 & 58.17 \\
\hline
\end{tabular}
\caption{Anchos de canal dimensionados para el OAI22+inv}
\end{center}
\end{table}

\begin{figure}[htbp]
  \centering
    \includegraphics[scale=0.5]{./Cicuito_Electric.png}
    \rule{35em}{0.5pt}
  \caption[IdealvsSim]{Compuerta \textit{Circuito simulado en el eléctric}}
  \label{fig:circuito_E}
\end{figure}

\begin{figure}[htbp]
\begin{center}
    \includegraphics[scale=0.5]{./Mejor_caso_red_nmos.png}
    \rule{35em}{0.5pt}
  \caption[Captura]{Gráfica para mejor caso del NMOS}
  \label{fig:CircuitoFO2}
  \end{center}
\end{figure}

\begin{figure}[htbp]
\begin{center}
    \includegraphics[scale=0.5]{./Peor_caso_red_nmos.png}
    \rule{35em}{0.5pt}
  \caption[Captura]{Gráfica para peor caso del NMOS}
  \label{fig:CircuitoFO2}
  \end{center}
\end{figure}

\begin{figure}[htbp]
\begin{center}
    \includegraphics[scale=0.5]{./Mejor_caso_red_pmos.png}
    \rule{35em}{0.5pt}
  \caption[Captura]{Gráfica para mejor caso del PMOS}
  \label{fig:CircuitoFO2}
  \end{center}
\end{figure}


\begin{figure}[htbp]
\begin{center}
    \includegraphics[scale=0.5]{./Peor_caso_red_pmos.png}
    \rule{35em}{0.5pt}
  \caption[Captura]{Gráfica para peor caso del PMOS}
  \label{fig:CircuitoFO2}
  \end{center}
\end{figure}

\begin{table}\label{table:Tabla_propagación}
\begin{center}
\begin{tabular}{c||c||c}
  & Mejor tiempo (ps) & Peor tiempo (ps)\\
\hline
\hline
tr & 79.4 & 147.4 \\
tf & 58.6 & 129.5 \\
\hline
\end{tabular}
\caption{Tiempos de retardo de propagación simulados en Electric}
\end{center}
\end{table}


\section{Análisis de datos y resultados.}



\section{Conclusiones.}
\begin{itemize}

\end{itemize}

%----------------------------------------------------------------------------------------
\begin{thebibliography}{3}

\bibliographystyle{unsrtnat} % Use the "unsrtnat" BibTeX style for formatting the Bibliography

\bibitem[Wey(1999)]{Wey1999}
[1] N. Weste, D. Harris. 
\newblock {CMOS VLSI Design: A Circuits and Systems Perspective , 4 edition.}.
\newblock \emph{Boston: Addison-Wesley}, 2010.

\bibitem[Wey(1999)]{Wey1999}
[2] J. Rabaey, A. Chandrakasan y B. Nikolic. 
\newblock { Digital Integrated Circuits: A Design Perspective.}.
\newblock \emph{Prentice Hall}, 2005.

\bibitem[Wey(1999)]{Wey1999}
[3]Test Data .On SemiconductorC5.Mosis. Recompilado de:
\newblock \emph{http://www.ie.itcr.ac.cr/achacon/ \\* Intro$\_$Diseno$\_$CI/Modelos$\_$Spice$\_$MOSIS/v03m-params.txt}, el 07/09/2015

\end{thebibliography}

\end{document}